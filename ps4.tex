\documentclass{article}
\usepackage[utf8]{inputenc}
\usepackage{enumitem}

\usepackage{amsfonts,amsthm,amsmath,amssymb}
\usepackage{array}
\usepackage{epsfig}
\usepackage{fullpage}
\usepackage{mathtools}
\usepackage{hyperref}


\input{macros.tex}

\title{Design and Analysis of Algorithms\\ {\bf Problem Set 4 (2st graded set)}}
\author{}
%\date{}

\usepackage{fullpage}

\begin{document}

\maketitle

\begin{quote}
    Write your name and computing ID here: Aobo Yang (ay6gv)
\end{quote}

\vspace{20px}

\begin{quote}
    Write your collaborators' computing ID here:
\end{quote}

\vspace{20px}

\begin{quote}
    Here write the pledge that you did not share any written material electronically or differently: I have neither given nor received aid on this assignment.
\end{quote}

\newpage
\paragraph{Problem 1:} In class, we proved that the height of every red-black tree is at most $O(\log n)$.

\begin{enumerate}[label=\alph*]
\item Where in that proof did we use property 5 of red-black trees? Briefly discuss that part of the proof just to convince the reader of the fact that you know how property 5 is playing a role there!

\item Show that if we remove property 5, the depths of red-black trees are no longer necessarily logarithmic.
\end{enumerate}

\paragraph{Answer:}
% Add your answer for problem 1 here:
\paragraph{}
a

Property 5 is used in the proof of the claim that the subtree at any node $x$ contains at least $2^{bh(x)}-1$ internal nodes. Here uses the definition of black-height, $bh(x)$, which means the number of black nodes on any simple path from a node $x$ but not including down to a leaf. This definition is built on Property 5 so that all descending simple paths from the node have the same number of black nodes. Without Property 5, the black-height of a node loses its meaning, because the number of top-down black nodes depends on which path we count. $bh(x)$ becomes a range of possible number instead of a constant. Furthermore, in the later induction proof of this claim, the conclusion that each child of node $x$ has a black-height of either $bh(x)$ or $bh(x)-1$ also relies on the same number of black nodes on each path. Therefore, this claim and its proof are based on Property 5.

\paragraph{}
b

For example, without Property 5, $n$ nodes can form a linear chain-like tree. In details, each node acts as the left child of another node and the right child of each node is $NIL$. All the nodes' color are black. Such tree can satisfy all the 4 other properties of red-black tree but the depths is $O(n)$ instead of $O(\log n)$.






\newpage
\paragraph{Problem 2:}
In class, we discussed the notion of universal hash functions. It was a \emph{family} $H$ of hash functions, and during the algorithm we will pick one from it. Suppose the universe of keys is $\set{0,\dots,100}$.

\begin{enumerate}[label=\alph*]
\item If we want to use the specific family of hash functions that in class we proved to be Universal, what is the smallest value of prime $p$ that we can use for this size of universe of keys?

\item Suppose the space of keys is all the nonnegative integer numbers up to $2^{100}-1$. (Note that writing down any of these numbers needs at most $100$ bits). Suppose we will not store all these numbers at the same time, but rather we will have a small set of size at most $1000$ numbers. For this problem, how big will you use the size of the array $T$ of the linked hashes?

\item For the setting of the problem in part (b), if we want to construct a family of universal hash functions, as we did in class, argue that the family will have more than $2^{200}$ hash functions in it. (Hint, we need to pick a prime that is bigger than $2^{100}-1$, and $2^{100}$ is not a prime!).

\item Explain briefly: how can we manage to work with such a huge family of hash functions while we cannot even write down the description of all such hash functions? (Hint, argue why we don't need to write down the description of those hash functions, and we can still use this family as we want.)
\end{enumerate}

\paragraph{Answer:}
% Add your answer for problem 2 here:
\paragraph{}
a

The smallest value is $101$, since the prime number $p$ needs ensure $p>k$, for all $k$ in the universe of keys.

\paragraph{}
b

I will use an array with size of $1000$. According to the Corollary 11.4 in the book, for an empty table with $m$ slot, it takes expected time $\Theta(n)$ to handle any sequence of $n$ INSERT, SEARCH, and DELETE operations containing $O(m)$ INSERT operations. Since the hash table needs to hold $1000$ numbers, we need $1000$ slots so that the hash table can handle these operations in linear time.

\paragraph{}
c

The hash function is

$$
h_{ab}(k)=((ak+b) \; mod \; p) \; mod \;m
$$

where $p$ is a prime number $p$ with $p>k$ for every possible key $k$.

The family of all such function can be represented as
$$
H_{pm}=\{ h_{ab} : a \in \mathbb{Z}^{*}_p ~ and ~ b \in \mathbb{Z}_p \}
$$

where
$$
\mathbb{Z}^{*}_p = \{1,2,\dots,p-1\}
$$
$$
\mathbb{Z}_p = \{0,1,\dots,p-1\}
$$

so

$$
|\mathbb{Z}^{*}_p| = p - 1
$$
$$
|\mathbb{Z}_p| = p
$$

Based on part (b), for every possible key $k$, we have

$$
k \leq 2^{100}-1
$$

while $2^{100}$ is not a prime number, so

$$
p >= 2^{100} + 1
$$
\newpage
\paragraph{Answer for problem 2 (if continued):}


\paragraph{}
Therefore,

$$\begin{aligned}
|H_{pm}| &= |\mathbb{Z}^{*}_p| * |\mathbb{Z}_p|\\
&=(p-1)p\\
&=2^{100}(2^{100}+1)\\
&>2^{200}
\end{aligned}$$

\paragraph{}
d

Because everytime when we need to choose a hash function, we only need to base on a large enough prime number to randomly generate the $2$ variables within the range: $a$ and $b$. Any such qualified pairs can produce a hash function of the family. Therefore, we don't need to write down the descriptions for all the family.

\newpage
\paragraph{Problem 3:}
For non-negative integers $m\leq n$, ${n \choose m}$ denotes the number of ways that we can choose $m$ objects from $\set{1,\dots,n}$ while the order of selection does not matter. For example ${n \choose 2} = \frac{n (n-1)}{2}$.
Pascal has a recursive relation for finding the values of ${n \choose m}$ as follows.
\begin{itemize}
    \item If $m > n$, then ${n \choose m}=0$.
    \item If $m =n$ or $m=0$, then ${n \choose m}=1$.
    \item Otherwise, ${n \choose m} = {n-1 \choose m} + {n-1 \choose m-1}$.
\end{itemize}


\begin{enumerate}[label=\alph*]
\item Argue very briefly, why Pascal's recursive relation is correct.
\item Write a (naiive) recursive program based on Pascal's recursive relation that computes the number.
\item Analyze the running time of your recursive implementation.
\item How long does the recursive program take to compute ${n \choose n/2}$? Is it a polynomial time over $n$? Hint, you can use the inequality $(\frac{n}{k})^k \leq {n \choose k}$ (No need to write a proof for the hint, but try to prove it yourself).
\item Write a bottom-up dynamic programming version of the same recursive algorithm.
\item What is the running time of the programs you wrote in previous 2 steps? (Based on $m,n$.)
\end{enumerate}


\paragraph{Answer:}
% Add your answer for problem 3 here:
\paragraph{}
a

If $m>n$, there is no way we can choose $m$ objects only from $n$ projects, so ${n \choose m}=0$

If $m=n$, there is only one way to choose all objects, so ${n \choose m}=0$

If $m=0$, there is only one way to choose no objects, so ${n \choose m}=0$

Otherwise,

$$\begin{aligned}
{n-1 \choose m} + {n-1 \choose m-1} &= \frac{(n-1)(n-2)\dots(n-m)}{m(m-1)\dots2} + \frac{(n-1)(n-2)\dots(n-m+1)}{(m-1)(m-2)\dots2}\\
&=\frac{(n-1)(n-2)\dots(n-m+1)(n-m)}{m(m-1)\dots2} + \frac{m(n-1)(n-2)\dots(n-m+1)}{m(m-1)(m-2)\dots2}\\
&=\frac{(n-1)(n-2)\dots(n-m+1)(n-m+m)}{m(m-1)\dots2}\\
&=\frac{n(n-1)(n-2)\dots(n-m+1)}{m(m-1)\dots2}\\
&={n \choose m}
\end{aligned}$$

So Pascal's recursive relation is correct.

\paragraph{}

b

\begin{verbatim}
    function pascalRec(n, m) {
        if (m > n) return 0;
        if (m == n || m == 0) return 1;

        return pascalRec(n-1, m) + pascalRec(n-1, m-1);
    }
\end{verbatim}

\paragraph{}

c


\newpage
\paragraph{Answer for problem 3 (if continued):}

\end{document}
